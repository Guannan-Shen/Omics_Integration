\documentclass[]{article}
\usepackage{lmodern}
\usepackage{amssymb,amsmath}
\usepackage{ifxetex,ifluatex}
\usepackage{fixltx2e} % provides \textsubscript
\ifnum 0\ifxetex 1\fi\ifluatex 1\fi=0 % if pdftex
  \usepackage[T1]{fontenc}
  \usepackage[utf8]{inputenc}
\else % if luatex or xelatex
  \ifxetex
    \usepackage{mathspec}
  \else
    \usepackage{fontspec}
  \fi
  \defaultfontfeatures{Ligatures=TeX,Scale=MatchLowercase}
\fi
% use upquote if available, for straight quotes in verbatim environments
\IfFileExists{upquote.sty}{\usepackage{upquote}}{}
% use microtype if available
\IfFileExists{microtype.sty}{%
\usepackage{microtype}
\UseMicrotypeSet[protrusion]{basicmath} % disable protrusion for tt fonts
}{}
\usepackage[margin=1in]{geometry}
\usepackage{hyperref}
\hypersetup{unicode=true,
            pdftitle={tidi\_MIBI EDA},
            pdfauthor={Guannan Shen},
            pdfborder={0 0 0},
            breaklinks=true}
\urlstyle{same}  % don't use monospace font for urls
\usepackage{color}
\usepackage{fancyvrb}
\newcommand{\VerbBar}{|}
\newcommand{\VERB}{\Verb[commandchars=\\\{\}]}
\DefineVerbatimEnvironment{Highlighting}{Verbatim}{commandchars=\\\{\}}
% Add ',fontsize=\small' for more characters per line
\usepackage{framed}
\definecolor{shadecolor}{RGB}{248,248,248}
\newenvironment{Shaded}{\begin{snugshade}}{\end{snugshade}}
\newcommand{\KeywordTok}[1]{\textcolor[rgb]{0.13,0.29,0.53}{\textbf{#1}}}
\newcommand{\DataTypeTok}[1]{\textcolor[rgb]{0.13,0.29,0.53}{#1}}
\newcommand{\DecValTok}[1]{\textcolor[rgb]{0.00,0.00,0.81}{#1}}
\newcommand{\BaseNTok}[1]{\textcolor[rgb]{0.00,0.00,0.81}{#1}}
\newcommand{\FloatTok}[1]{\textcolor[rgb]{0.00,0.00,0.81}{#1}}
\newcommand{\ConstantTok}[1]{\textcolor[rgb]{0.00,0.00,0.00}{#1}}
\newcommand{\CharTok}[1]{\textcolor[rgb]{0.31,0.60,0.02}{#1}}
\newcommand{\SpecialCharTok}[1]{\textcolor[rgb]{0.00,0.00,0.00}{#1}}
\newcommand{\StringTok}[1]{\textcolor[rgb]{0.31,0.60,0.02}{#1}}
\newcommand{\VerbatimStringTok}[1]{\textcolor[rgb]{0.31,0.60,0.02}{#1}}
\newcommand{\SpecialStringTok}[1]{\textcolor[rgb]{0.31,0.60,0.02}{#1}}
\newcommand{\ImportTok}[1]{#1}
\newcommand{\CommentTok}[1]{\textcolor[rgb]{0.56,0.35,0.01}{\textit{#1}}}
\newcommand{\DocumentationTok}[1]{\textcolor[rgb]{0.56,0.35,0.01}{\textbf{\textit{#1}}}}
\newcommand{\AnnotationTok}[1]{\textcolor[rgb]{0.56,0.35,0.01}{\textbf{\textit{#1}}}}
\newcommand{\CommentVarTok}[1]{\textcolor[rgb]{0.56,0.35,0.01}{\textbf{\textit{#1}}}}
\newcommand{\OtherTok}[1]{\textcolor[rgb]{0.56,0.35,0.01}{#1}}
\newcommand{\FunctionTok}[1]{\textcolor[rgb]{0.00,0.00,0.00}{#1}}
\newcommand{\VariableTok}[1]{\textcolor[rgb]{0.00,0.00,0.00}{#1}}
\newcommand{\ControlFlowTok}[1]{\textcolor[rgb]{0.13,0.29,0.53}{\textbf{#1}}}
\newcommand{\OperatorTok}[1]{\textcolor[rgb]{0.81,0.36,0.00}{\textbf{#1}}}
\newcommand{\BuiltInTok}[1]{#1}
\newcommand{\ExtensionTok}[1]{#1}
\newcommand{\PreprocessorTok}[1]{\textcolor[rgb]{0.56,0.35,0.01}{\textit{#1}}}
\newcommand{\AttributeTok}[1]{\textcolor[rgb]{0.77,0.63,0.00}{#1}}
\newcommand{\RegionMarkerTok}[1]{#1}
\newcommand{\InformationTok}[1]{\textcolor[rgb]{0.56,0.35,0.01}{\textbf{\textit{#1}}}}
\newcommand{\WarningTok}[1]{\textcolor[rgb]{0.56,0.35,0.01}{\textbf{\textit{#1}}}}
\newcommand{\AlertTok}[1]{\textcolor[rgb]{0.94,0.16,0.16}{#1}}
\newcommand{\ErrorTok}[1]{\textcolor[rgb]{0.64,0.00,0.00}{\textbf{#1}}}
\newcommand{\NormalTok}[1]{#1}
\usepackage{longtable,booktabs}
\usepackage{graphicx,grffile}
\makeatletter
\def\maxwidth{\ifdim\Gin@nat@width>\linewidth\linewidth\else\Gin@nat@width\fi}
\def\maxheight{\ifdim\Gin@nat@height>\textheight\textheight\else\Gin@nat@height\fi}
\makeatother
% Scale images if necessary, so that they will not overflow the page
% margins by default, and it is still possible to overwrite the defaults
% using explicit options in \includegraphics[width, height, ...]{}
\setkeys{Gin}{width=\maxwidth,height=\maxheight,keepaspectratio}
\IfFileExists{parskip.sty}{%
\usepackage{parskip}
}{% else
\setlength{\parindent}{0pt}
\setlength{\parskip}{6pt plus 2pt minus 1pt}
}
\setlength{\emergencystretch}{3em}  % prevent overfull lines
\providecommand{\tightlist}{%
  \setlength{\itemsep}{0pt}\setlength{\parskip}{0pt}}
\setcounter{secnumdepth}{5}
% Redefines (sub)paragraphs to behave more like sections
\ifx\paragraph\undefined\else
\let\oldparagraph\paragraph
\renewcommand{\paragraph}[1]{\oldparagraph{#1}\mbox{}}
\fi
\ifx\subparagraph\undefined\else
\let\oldsubparagraph\subparagraph
\renewcommand{\subparagraph}[1]{\oldsubparagraph{#1}\mbox{}}
\fi

%%% Use protect on footnotes to avoid problems with footnotes in titles
\let\rmarkdownfootnote\footnote%
\def\footnote{\protect\rmarkdownfootnote}

%%% Change title format to be more compact
\usepackage{titling}

% Create subtitle command for use in maketitle
\providecommand{\subtitle}[1]{
  \posttitle{
    \begin{center}\large#1\end{center}
    }
}

\setlength{\droptitle}{-2em}

  \title{tidi\_MIBI EDA}
    \pretitle{\vspace{\droptitle}\centering\huge}
  \posttitle{\par}
    \author{Guannan Shen}
    \preauthor{\centering\large\emph}
  \postauthor{\par}
      \predate{\centering\large\emph}
  \postdate{\par}
    \date{April 16, 2019}


\begin{document}
\maketitle

{
\setcounter{tocdepth}{5}
\tableofcontents
}
\begin{verbatim}
## [1] "/home/guanshim/Documents/gitlab/tidi_MIBI/Code"
\end{verbatim}

\subsection{Data}\label{data}

\begin{verbatim}
## Loading required package: pacman
\end{verbatim}

\begin{longtable}[]{@{}l@{}}
\toprule
Levels of Microbiome Data\tabularnewline
\midrule
\endhead
Phylum\tabularnewline
Class\tabularnewline
Order\tabularnewline
Family\tabularnewline
Genus\tabularnewline
Species\tabularnewline
\bottomrule
\end{longtable}

\subsubsection{Family Level Microbiome and Clinical
Data}\label{family-level-microbiome-and-clinical-data}

\begin{longtable}[]{@{}llllll@{}}
\toprule
V1 & V2 & V3 & V4 & V5 & V6\tabularnewline
\midrule
\endhead
OTU\_Name & MIHIV119B & MIHIV124B & MIHIV132B & MIHIV138B &
MIHIV154B\tabularnewline
root & 31995 & 36683 & 81909 & 132706 & 108756\tabularnewline
Bacteria/Firmicutes/Clostridia/Clostridiales/Lachnospiraceae & 17076 &
47 & 76 & 36324 & 25508\tabularnewline
Bacteria/Bacteroidetes/Bacteroidia/Bacteroidales/Bacteroidaceae & 3029 &
49 & 32 & 34482 & 36017\tabularnewline
Bacteria/Bacteroidetes/Bacteroidia/Bacteroidales/Prevotellaceae & 1 & 41
& 453 & 4 & 17\tabularnewline
Bacteria/Firmicutes/Clostridia/Clostridiales/Ruminococcaceae & 3401 & 33
& 17 & 18636 & 7270\tabularnewline
\bottomrule
\end{longtable}

\begin{longtable}[]{@{}lllllrrrrl@{}}
\toprule
Lib & diagnosis & sample\_type & Sex & sex\_dx & Age & Decade & Over50 &
Over40 & BMI\tabularnewline
\midrule
\endhead
MIHIV119B & Neg & Biop & M & M\_Neg & 27 & 2 & 0 & 0 & OV\tabularnewline
MIHIV138B & Neg & Biop & M & M\_Neg & 29 & 2 & 0 & 0 & N\tabularnewline
MIHIV178B & Neg & Biop & M & M\_Neg & 33 & 3 & 0 & 0 & OV\tabularnewline
MIHIV255B & Neg & Biop & M & M\_Neg & 34 & 3 & 0 & 0 & OB\tabularnewline
MIHIV278B & Neg & Biop & M & M\_Neg & 23 & 2 & 0 & 0 & na\tabularnewline
MIHIV361B & Neg & Biop & F & F\_Neg & 33 & 3 & 0 & 0 & N\tabularnewline
\bottomrule
\end{longtable}

\subsubsection{Filtering the Microbiome
Data}\label{filtering-the-microbiome-data}

Prevalence cutoff: 5\% (i.e., at least 5\% of libaries must be
represented to keep OTU)\\
Relative abundance cutoff: 1\% (i.e., at least one library must have RA
\textgreater{} 1\% to keep OTU).

\begin{verbatim}
## No "Unclassified" category found: data unchangedPrevalence cutoff: 5% (i.e., at least 5% of libaries must be represented to keep OTU)
## Created new 'Other' Taxa in OTU table.
## Collapsed 3 OTUs to 'Other' in OTU table.
## Converted 39 counts to 'Other' in OTU table.
## Remaining OTUs: 15  (Including 'Other').
## 
## Relative abundance cutoff: 1 % (i.e., at least one library must have RA > 1 % to keep OTU).
## Created new "Other" category.
## Collapsed 5 OTUs to "Other" otu category.
## Converted 2125 counts to "Other" otu category.
## Remaining OTUS: 11  (Including "Other").
## 
## Found "Unclassified" category in input data
## Created new "Other" category.
## Converted 36185 counts to "Other" otu category.
## Remaining OTUS: 66  (Including "Other").
## 
## Prevalence cutoff: 5% (i.e., at least 5% of libaries must be represented to keep OTU)
## Found 'Other' category in input data.
## Collapsed 16 OTUs to 'Other' in OTU table.
## Converted 552 counts to 'Other' in OTU table.
## Remaining OTUs: 50  (Including 'Other').
## 
## Relative abundance cutoff: 1 % (i.e., at least one library must have RA > 1 % to keep OTU).
## Found "Other" category in input data.
## Collapsed 18 OTUs to "Other" otu category.
## Converted 11541 counts to "Other" otu category.
## Remaining OTUS: 32  (Including "Other").
## 
## Found "Unclassified" category in input data
## Created new "Other" category.
## Converted 36185 counts to "Other" otu category.
## Remaining OTUS: 131  (Including "Other").
## 
## Prevalence cutoff: 5% (i.e., at least 5% of libaries must be represented to keep OTU)
## Found 'Other' category in input data.
## Collapsed 30 OTUs to 'Other' in OTU table.
## Converted 756 counts to 'Other' in OTU table.
## Remaining OTUs: 101  (Including 'Other').
## 
## Relative abundance cutoff: 1 % (i.e., at least one library must have RA > 1 % to keep OTU).
## Found "Other" category in input data.
## Collapsed 48 OTUs to "Other" otu category.
## Converted 21423 counts to "Other" otu category.
## Remaining OTUS: 53  (Including "Other").
## 
## Found "Unclassified" category in input data
## Created new "Other" category.
## Converted 36185 counts to "Other" otu category.
## Remaining OTUS: 270  (Including "Other").
## 
## Prevalence cutoff: 5% (i.e., at least 5% of libaries must be represented to keep OTU)
## Found 'Other' category in input data.
## Collapsed 64 OTUs to 'Other' in OTU table.
## Converted 1339 counts to 'Other' in OTU table.
## Remaining OTUs: 206  (Including 'Other').
## 
## Relative abundance cutoff: 1 % (i.e., at least one library must have RA > 1 % to keep OTU).
## Found "Other" category in input data.
## Collapsed 122 OTUs to "Other" otu category.
## Converted 44812 counts to "Other" otu category.
## Remaining OTUS: 84  (Including "Other").
## 
## Contains 31 subjects/libraries from Explicet OTU file.
\end{verbatim}

\subsubsection{Combined Dataset}\label{combined-dataset}

Combined Dataset contains microbiome data at Phylum, Order, Family and
Genus levels, together with the clinical parameters.\\
Particularly, \emph{Lib} is the donor, library; \emph{Total} is the
library size, the total counts per donor; \emph{cts} is the counts per
the library and taxa combination; \emph{clr} is the centered log-ratio
transformation of the relative abundance; \emph{per} is the relative
abundance percentage. \emph{diagnosis} is HIV-positive or HIV-negative.

\begin{verbatim}
## [1] 5580  115
\end{verbatim}

\begin{longtable}[]{@{}lllrrrrrllllrrr@{}}
\toprule
Rank & Lib & Taxa & Total & bin & cts & clr & per & diagnosis &
sample\_type & Sex & sex\_dx & Age & Decade & Over50\tabularnewline
\midrule
\endhead
Phylum & MIHIV119B & Firmicutes & 31995 & 1 & 21630 & 10.612186 &
67.604313 & Neg & Biop & M & M\_Neg & 27 & 2 & 0\tabularnewline
Phylum & MIHIV124B & Firmicutes & 36683 & 1 & 21727 & 12.865424 &
59.229071 & Pos & Biop & M & M\_Pos & 48 & 4 & 0\tabularnewline
Phylum & MIHIV132B & Firmicutes & 81909 & 1 & 2881 & 7.706506 & 3.517318
& Pos & Biop & M & M\_Pos & 25 & 2 & 0\tabularnewline
Phylum & MIHIV138B & Firmicutes & 132706 & 1 & 87480 & 14.171150 &
65.920154 & Neg & Biop & M & M\_Neg & 29 & 2 & 0\tabularnewline
Phylum & MIHIV154B & Firmicutes & 108756 & 1 & 58615 & 14.885815 &
53.895877 & Pos & Biop & F & F\_Pos & 58 & 5 & 1\tabularnewline
Phylum & MIHIV178B & Firmicutes & 105542 & 1 & 43326 & 14.613114 &
41.050956 & Neg & Biop & M & M\_Neg & 33 & 3 & 0\tabularnewline
\bottomrule
\end{longtable}

\subsection{Subset of Data for Downstream
Analysis}\label{subset-of-data-for-downstream-analysis}

Using only the family level data.

\begin{verbatim}
## Found "Unclassified" category in input data
## Created new "Other" category.
## Converted 58364 counts to "Other" otu category.
## Remaining OTUS: 53  (Including "Other").
## 
## Prevalence cutoff: 5% (i.e., at least 5% of libaries must be represented to keep OTU)
## No OTUs with prevalence < 5Relative abundance cutoff: 1 % (i.e., at least one library must have RA > 1 % to keep OTU).
## No OTUs with max relative abundance < 1 % so no "Other" category created.
\end{verbatim}

\begin{longtable}[]{@{}lllrrrrrllllrrr@{}}
\toprule
Rank & Lib & Taxa & Total & bin & cts & clr & per & diagnosis &
sample\_type & Sex & sex\_dx & Age & Decade & Over50\tabularnewline
\midrule
\endhead
Family & MIHIV119B & Actin:Actin:Actinomycetaceae & 31995 & 0 & 0 &
-7.385511 & 0 & Neg & Biop & M & M\_Neg & 27 & 2 & 0\tabularnewline
Family & MIHIV124B & Actin:Actin:Actinomycetaceae & 36683 & 0 & 0 &
-10.471096 & 0 & Pos & Biop & M & M\_Pos & 48 & 4 & 0\tabularnewline
Family & MIHIV132B & Actin:Actin:Actinomycetaceae & 81909 & 0 & 0 &
-8.933086 & 0 & Pos & Biop & M & M\_Pos & 25 & 2 & 0\tabularnewline
Family & MIHIV138B & Actin:Actin:Actinomycetaceae & 132706 & 0 & 0 &
-9.194577 & 0 & Neg & Biop & M & M\_Neg & 29 & 2 & 0\tabularnewline
Family & MIHIV154B & Actin:Actin:Actinomycetaceae & 108756 & 0 & 0 &
-9.631099 & 0 & Pos & Biop & F & F\_Pos & 58 & 5 & 1\tabularnewline
Family & MIHIV178B & Actin:Actin:Actinomycetaceae & 105542 & 0 & 0 &
-8.426626 & 0 & Neg & Biop & M & M\_Neg & 33 & 3 & 0\tabularnewline
\bottomrule
\end{longtable}

\subsubsection{Alpha and Beta
Diversities}\label{alpha-and-beta-diversities}

We can calculate the alpha and beta diversities of our sample set as
well using the functions \emph{Alpha\_Div} and \emph{Beta\_Div},
respectively.

\emph{Alpha\_Div} calculates the Sobs, Chao1, Goods, ShannonE, ShannonH,
and SimpsonD alpha diversities and attaches them as columns to the
original tidy\_MIBI set given. It can handle any combination of taxa
levels, and will calculate them all separately. The \emph{iter} argument
is the number of bootstrap iterations used in the calculations. Warning:
this can take several minutes to run.

\emph{Beta\_Div} calculates the Bray-Curtis and Morisita-Horn beta
diversities. It requires you to specify the rank of taxa calculated, and
will return the two beta diversity matrices in a list.

\begin{Shaded}
\begin{Highlighting}[]
\NormalTok{## Alpha Diversity}
\NormalTok{all_fam_alpha <-}\StringTok{ }\KeywordTok{Alpha_Div}\NormalTok{(all_fam, }\DataTypeTok{rank =} \StringTok{"Family"}\NormalTok{) ## Not run for time}
\NormalTok{all_fam_alpha}

\NormalTok{## Beta Diversity}
\NormalTok{## vegdist \{vegan\}}
\NormalTok{all_fam_beta <-}\StringTok{ }\KeywordTok{Beta_Div}\NormalTok{(all_fam, }\DataTypeTok{rank =} \StringTok{"Family"}\NormalTok{)}
\CommentTok{# all_fam_beta}

\CommentTok{# all_fam_beta$MH ## Morisita-Horn}
\CommentTok{# all_fam_beta$BC ## Bray-Curtis }
\end{Highlighting}
\end{Shaded}

\subsubsection{EDA with Bar Chart}\label{eda-with-bar-chart}

\begin{Shaded}
\begin{Highlighting}[]
\NormalTok{all_fam }\OperatorTok\StringTok{ }\KeywordTok{NB_Bars}\NormalTok{(diagnosis,                ## Covariate of interest}
                   \DataTypeTok{top_taxa =} \DecValTok{10}\NormalTok{,         ## How many named taxa we want}
                   \DataTypeTok{RA =} \DecValTok{0}\NormalTok{,               ## Only need one of top_taxa / RA specified}
                   \DataTypeTok{specific_taxa =} \OtherTok{NULL}\NormalTok{, ## No specific taxa of interest}
                   \DataTypeTok{xlab =} \StringTok{"HIV Status"}\NormalTok{, }\DataTypeTok{main =} \StringTok{"Title"}\NormalTok{, }\DataTypeTok{subtitle =} \StringTok{"Sub"}\NormalTok{) ## Labels}
\end{Highlighting}
\end{Shaded}

\subsubsection{Negative Binomial Models}\label{negative-binomial-models}

It is standard to model the relative abundance of each taxa using a
negative binomial distribution. The function \emph{NB\_mods} will create
negative binomial models for each taxa within a specified rank using the
observed counts and the Total as an offset. It uses \emph{glm.nb} from
the \emph{MASS} package to fit these models.

\emph{NB\_mods} will take each variable you specify as a new term to add
into the model. For instance, if we include \emph{diagnosis, Age, Sex},
the function will run the model
\[log(\hat{cts}) = \beta_0  + \beta_1 diagnosis + \beta_2 Age + \beta_3 Sex + log(Total).\]
You can also include interaction terms (such as Age*Sex) as you can with
any other model.

25 taxa converged. 28 taxa did not converge.

\begin{Shaded}
\begin{Highlighting}[]
\NormalTok{## run in a seperate r script is fast  }
\KeywordTok{dim}\NormalTok{(all_fam)}
\end{Highlighting}
\end{Shaded}

\begin{verbatim}
## [1] 1643   60
\end{verbatim}

\begin{Shaded}
\begin{Highlighting}[]
\NormalTok{nb_fam <-}\StringTok{ }\NormalTok{all_fam }\OperatorTok\StringTok{      }\NormalTok{## tidy_MIBI set}
\StringTok{  }\KeywordTok{NB_mods}\NormalTok{(}\DataTypeTok{rank =} \StringTok{"Family"}\NormalTok{, ## Rank of taxa we want to model}
\NormalTok{          diagnosis, Age, Sex       ## The covariates in our model (Group + EWG)}
\NormalTok{          )}
\end{Highlighting}
\end{Shaded}

\begin{verbatim}
## 25 taxa converged.
## 28 taxa did not converge.
\end{verbatim}

\begin{Shaded}
\begin{Highlighting}[]
\NormalTok{                           ## If we wanted the covariates to be Group+EWG+Group*EWG}
\NormalTok{                           ## we could just input Group*EWG in the covariate spot.}
\end{Highlighting}
\end{Shaded}

\begin{Shaded}
\begin{Highlighting}[]
\CommentTok{# load( }
\CommentTok{#      file = "C:/Users/hithr/Documents/Stats/gitlab/tidi_MIBI/DataProcessed/nb_model/nb_fam_diagnosis_age_sex.RData" )}

\NormalTok{nb_fam}\OperatorTok{$}\NormalTok{Convergent_Summary }\OperatorTok\StringTok{ }\KeywordTok{filter}\NormalTok{(Coef }\OperatorTok{==}\StringTok{ "diagnosisPos"}\NormalTok{, FDR_Pval }\OperatorTok{<}\StringTok{ }\FloatTok{0.05}\NormalTok{ ) }\OperatorTok\StringTok{ }\KeywordTok{arrange}\NormalTok{(P_val) }\OperatorTok\StringTok{ }\NormalTok{kable}
\end{Highlighting}
\end{Shaded}

\begin{longtable}[]{@{}llrlrrr@{}}
\toprule
Taxa & Coef & Beta & CI & Z & P\_val & FDR\_Pval\tabularnewline
\midrule
\endhead
Spiro:Spiro:Brachyspiraceae & diagnosisPos & 0.8560718 & (0.8062,
0.9063) & 33.5440 & 0.0000 & 0.0000\tabularnewline
Prote:Gamma:Xanthomonadaceae & diagnosisPos & 2.6489268 & (1.1732,
4.0606) & 3.6596 & 0.0003 & 0.0015\tabularnewline
Prote:Gamma:Moraxellaceae & diagnosisPos & 1.2557366 & (0.3371, 2.1336)
& 3.2107 & 0.0013 & 0.0060\tabularnewline
Prote:Alpha:Rhodospirillaceae & diagnosisPos & -2.7720809 & (-5.8443,
-0.3994) & -2.6118 & 0.0090 & 0.0346\tabularnewline
\bottomrule
\end{longtable}

\subsubsection{Stacked Bar Charts}\label{stacked-bar-charts}

Stacked bar charts of taxa RA is a very useful visualization for
microbiome research. We have created a function that will calculate the
estimated RA of all taxa based on the convergent models. This gives us
the ability to visualize stacked bar charts of RA while controlling for
other variables in the model. For instance, if our model is
\[log(\hat{cts}) = \beta_0  + \beta_1 diagnosis + \beta_2 Age + \beta_3 Sex + log(Total)\]
we can visualize the estimated differences in RA among different groups
while holding Age and Sex constant.

The function \emph{NB\_Bars} will create these stacked bar charts based
on the output of the \emph{NB\_mods} function. It requires the name of a
covariate in your model and can create plots based on main effects or
interactions. If a continuous variable is one of the supplied
covariates, \emph{NB\_Bars} requires a min and max value to plug into
the models for estimation.

\emph{NB\_Bars} gives you the ability to control how taxa are chosen to
be named and plotted or aggregated into thYou can specify how many taxa
will be named and used in the stacked bar charts using \emph{top\_taxa}.
This function will take X taxa with the highest RA and put everything
else into the ``Other'' category. You can also specify the \emph{RA}
option and it will take all taxa below the RA cutoff you set and put it
into the ``Other'' category. If a particular taxa is of interest you can
use \emph{specific\_taxa} to pull any taxa out of the ``Other'' category
if it doesn't meet the bar set by either \emph{top\_taxa} or \emph{RA}.

\begin{Shaded}
\begin{Highlighting}[]
\NormalTok{nb_fam }\OperatorTok\StringTok{ }\KeywordTok{NB_Bars}\NormalTok{(diagnosis,                ## Covariate of interest}
                   \DataTypeTok{top_taxa =} \DecValTok{10}\NormalTok{,         ## How many named taxa we want}
                   \DataTypeTok{RA =} \DecValTok{0}\NormalTok{,               ## Only need one of top_taxa / RA specified}
                   \DataTypeTok{specific_taxa =} \OtherTok{NULL}\NormalTok{, ## No specific taxa of interest}
                   \DataTypeTok{xlab =} \StringTok{"HIV Status"}\NormalTok{, }\DataTypeTok{main =} \StringTok{"Title"}\NormalTok{, }\DataTypeTok{subtitle =} \StringTok{"Sub"}\NormalTok{) ## Labels}
\end{Highlighting}
\end{Shaded}

\subsection{Test of the Package}\label{test-of-the-package}

\begin{enumerate}
\def\labelenumi{\arabic{enumi}.}
\tightlist
\item
  No ``Unclassified'' category found: data unchangedPrevalence cutoff:
  5\% (i.e., at least 5\% of libaries must be represented to keep OTU).
  Should have a new line here.\\
\item
  In the otu\_filter or NB\_mods, can rank equal to a vector of levels,
  not just one level.\\
\item
  The alpha and beta diversities calculation function still require the
  \emph{rank} argument, while the rank has already been specified in
  previous otu\_filter.\\
\item
  all\_fam\_beta\$MH is not callable.\\
\item
  all\_fam\_beta\$BC is not callable.\\
\item
  The log(Total) in the negative binomial model.\\
\item
  nb\_fam\$Convergent\_Summary better have non-zero p value.\\
\item
  Variable specified for bar charts is not in original model. for
  NB\_Bars.
\end{enumerate}


\end{document}
